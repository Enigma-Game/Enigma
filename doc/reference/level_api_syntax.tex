\documentclass[10pt,landscape]{article}
\usepackage{multicol}
\usepackage{calc}
\usepackage{ifthen}
\usepackage[landscape]{geometry}

\geometry{top=1cm,left=1cm,right=1cm,bottom=1cm}
\pagestyle{empty}

% Redefine section commands to use less space
\makeatletter
\renewcommand{\section}{\@startsection{section}{1}{0mm}%
                                {-1ex plus -.5ex minus -.2ex}%
                                {0.5ex plus .2ex}%
                                {\normalfont\large\bfseries}}
\renewcommand{\subsection}{\@startsection{subsection}{2}{0mm}%
                                {-1explus -.5ex minus -.2ex}%
                                {0.5ex plus .2ex}%
                                {\normalfont\normalsize\bfseries}}
\renewcommand{\subsubsection}{\@startsection{subsubsection}{3}{0mm}%
                                {-1ex plus -.5ex minus -.2ex}%
                                {1ex plus .2ex}%
                                {\normalfont\small\bfseries}}
\makeatother

\begin{document}

\raggedright
\footnotesize
\begin{multicols}{3}
\setlength{\premulticols}{1pt}
\setlength{\postmulticols}{1pt}
\setlength{\multicolsep}{1pt}
\setlength{\columnsep}{2pt}

\begin{center}
\Large{\textbf{Enigma Level API II}} \\
\large{\textbf{Syntax Sheet}}
\end{center}

\begin{center}
\rule{0.6\linewidth}{0.25pt}
\end{center}

On syntax descriptions of datatype operators or methods we need to list allowed argument types.
Often several types are possible and you are allowed to choose any of a list. In these cases we
enlist the types enclosed by \verb!<! and \verb!>! and separated by \verb!|!. These
characters are not part of the operator or method itself and should thus not be typed into
the level code. Note that we keep square braces \verb![!, \verb!]! and curly braces \verb!{!, \verb!}!
as literal Lua symbols. When these braces appear in the syntax you need to type them in the code.

\section*{Types}

\begin{description}
  \item[position]A position within the world that can be described by an x and y coordinate.
  \item[positions]The singleton type of the repository of all named positions.
  \item[object]An Enigma object like a stone, item, floor, other. Any object is a position, too.
  \item[group]A list of objects.
  \item[namedobjects]The singleton type of the repository of all named objects.
  \item[default]The singleton type of default values that can be used instead of Lua’s ‘nil’ in anonymous table tile definitions.
  \item[tile]A description of one or several objects for a common grid position (floor, item, stone, actor)
  \item[tiles]The singleton type of the repository of all tile instances.
  \item[world]The singleton type of the world that contains all objects.
  \item[position list]A list of positions.
\end{description}

\begin{center}
\rule{0.6\linewidth}{0.25pt}
\end{center}


\section*{Position}

\subsection*{Addition/Subtraction}
\begin{verbatim}
result = pos <+|-> <pos | obj | cpos | polist>
result = <pos | obj | cpos | polist> <+|-> pos
\end{verbatim}

\subsection*{Multiplication/Division}
\begin{verbatim}
result = pos <*|/> number
result = number * pos
\end{verbatim}

\subsection*{Sign}
\begin{verbatim}
result = -pos
\end{verbatim}

\subsection*{Center}
\begin{verbatim}
result = #pos
\end{verbatim}

\subsection*{Comparison}
\begin{verbatim}
result = pos1 <==|~=> pos2
\end{verbatim}

\subsection*{Concatenation}
\begin{verbatim}
result = pos1 .. <pos2 | polist>
result = <pos1 | polist> .. pos2
\end{verbatim}

\subsection*{Coordinate Access}
\begin{verbatim}
result = pos["x"]
result = pos["y"]
result1, result2 = pos:xy()
\end{verbatim}

\subsection*{Grid Rounding}
\begin{verbatim}
result = pos:grid()
\end{verbatim}

\subsection*{Existence}
\begin{verbatim}
result = pos:exists()
\end{verbatim}

\subsection*{Norm}
\begin{verbatim}
result = pos:norm()
\end{verbatim}


\section*{Object}

\subsection*{Attribute Access}
\begin{verbatim}
result = obj["attributename"]
obj["attributename"] = value
obj:set({attributename1=value1,
         attributename2=value2, ...})
\end{verbatim}

\subsection*{Messaging}
\begin{verbatim}
result = obj:message("msg", value)
result = obj:msg(value)
\end{verbatim}

\subsection*{Comparison}
\begin{verbatim}
result = obj1 <==|~=> obj2
\end{verbatim}

\subsection*{Existence}
\begin{verbatim}
result = -obj
result = obj:exists()
\end{verbatim}

\subsection*{Kill}
\begin{verbatim}
obj:kill()
\end{verbatim}

\subsection*{Kind Checks}
\begin{verbatim}
result = obj:is("kind")
result = obj:kind()
\end{verbatim}

\subsection*{Coordinate Access}
\begin{verbatim}
result = obj["x"]
result = obj["y"]
result1, result2 = obj:xy()
\end{verbatim}

\subsection*{Addition/Subtraction}
\begin{verbatim}
result = obj <+|-> <pos | obj | cpos | polist>
result = <pos | obj | cpos | polist> <+|-> obj
\end{verbatim}

\subsection*{Center}
\begin{verbatim}
result = #obj
\end{verbatim}

\subsection*{Join}
\begin{verbatim}
result = obj + group
result = group + obj
\end{verbatim}

\subsection*{Intersection}
\begin{verbatim}
result = obj * group
result = group * obj
\end{verbatim}

\subsection*{Difference}
\begin{verbatim}
result = obj - group
result = group - obj
\end{verbatim}

\subsection*{Sound}
\begin{verbatim}
result = obj:sound("name", volume)
\end{verbatim}



\section*{Group}

\subsection*{Messaging}
\begin{verbatim}
result = group:message("msg", value)
result = group:msg(value)
\end{verbatim}

\subsection*{Attribute Write}
\begin{verbatim}
group["attributename"] = value
group:set({attributename1=value1,
           attributename2=value2, ...})
\end{verbatim}

\subsection*{Comparison}
\begin{verbatim}
result = group1 <==|~=> group2
\end{verbatim}

\subsection*{Length/Size}
\begin{verbatim}
result = #group
\end{verbatim}

\subsection*{Member Access}
\begin{verbatim}
result = group[index]
result = group[obj]
\end{verbatim}

\subsection*{Loop}
\begin{verbatim}
for obj in group do ... end
\end{verbatim}

\subsection*{Join}
\begin{verbatim}
result = group + <obj | group>
result = <obj | group> + group
\end{verbatim}

\subsection*{Intersection}
\begin{verbatim}
result = <obj | group> * group
result = group * <obj | group>
\end{verbatim}

\subsection*{Difference}
\begin{verbatim}
result = <obj | group> - group
result = group - <obj | group>
\end{verbatim}

\subsection*{Shuffle}
\begin{verbatim}
result = group:shuffle()
\end{verbatim}

\subsection*{Sorting}
\begin{verbatim}
result = group:sort("circular")
result = group:sort("linear" <, direction>)
result = group:sort()
\end{verbatim}

\subsection*{Subset}
\begin{verbatim}
result = group:sub(number)
result = group:sub(start, end)
result = group:sub(start, -number)
\end{verbatim}

\subsection*{Nearest Object}
\begin{verbatim}
result = group:nearest(obj)
\end{verbatim}



\section*{NamedObjects}

\subsection*{Repository Request}
\begin{verbatim}
result = no["name"]
\end{verbatim}

\subsection*{Object Naming}
\begin{verbatim}
no["name"] = obj
\end{verbatim}



\section*{PositionList}

\subsection*{Comparison}
\begin{verbatim}
result = polist1 <==|~=> polist2
\end{verbatim}

\subsection*{Length}
\begin{verbatim}
result = #polist
\end{verbatim}

\subsection*{Member Access}
\begin{verbatim}
result = group[index]
\end{verbatim}

\subsection*{Concatenation}
\begin{verbatim}
result = polist1 .. <pos | polist2>
result = <pos | polist1> .. polist2
\end{verbatim}

\subsection*{Translation}
\begin{verbatim}
result = polist <+|-> <pos | obj | cpos>
result = <pos | obj | cpos> <+|-> polist
\end{verbatim}

\subsection*{Stretching}
\begin{verbatim}
result = polist * number
result = number * polist
\end{verbatim}



\section*{Positions Repository}

\subsection*{Repository Request}
\begin{verbatim}
result = po["name"]
\end{verbatim}

\subsection*{Repository Storage}
\begin{verbatim}
po["name"] = obj
\end{verbatim}

\subsection*{Position Convertion}
\begin{verbatim}
result = po(<obj | pos | {x, y} | x,y>)
\end{verbatim}

\subsection*{PositionList Convertion}
\begin{verbatim}
result = po(group | {pos1, pos2, pos3 })
\end{verbatim}



\section*{Tile and Object Declaration}

\subsection*{Tile concat}
\begin{verbatim}
result = tile .. <tile | odecl>
result = <tile | odecl> .. tile
\end{verbatim}



\section*{Tiles Repository}

\subsection*{ Tiles Storage}
\begin{verbatim}
ti["key"] = <tile | odecl>
\end{verbatim}

\subsection*{Tiles Request}
\begin{verbatim}
result = ti["key"]
\end{verbatim}

\subsection*{ Tile Convertion}
\begin{verbatim}
result = ti(odecl)
\end{verbatim}



\section*{World}

\subsection*{World Creation}
\begin{verbatim}
width, height = wo(topresolver, defaultkey, map)
width, height = wo(topresolver, libmap)
width, height = wo(topresolver, defaultkey, width, height)
\end{verbatim}

\subsection*{World Tile Set}
\begin{verbatim}
wo[<object | position | table |
    group | polist>] = tile_declarations
\end{verbatim}

\subsection*{Global Attribute Set}
\begin{verbatim}
wo["attritbutename"] = value
\end{verbatim}

\subsection*{Global Attribute Get}
\begin{verbatim}
var = wo["attritbutename"]
\end{verbatim}

\subsection*{add}
\begin{verbatim}
wo:add(tile_declarations)
wo:add(target, tile_declarations)
\end{verbatim}

\subsection*{drawBorder}
\begin{verbatim}
wo:drawBorder(upperleft_edge, lowerright_edge,
              <tile | key, resolver>)
wo:drawBorder(upperleft_edge, width, height,
              <tile | key, resolver>)
\end{verbatim}

\subsection*{drawMap}
\begin{verbatim}
wo:drawMap(resolver, anchor, ignore, map, [readdir])
wo:drawMap(resolver, anchor, libmap_map, [readdir])
\end{verbatim}

\subsection*{drawRect}
\begin{verbatim}
wo:drawRect(upperleft_edge, lowerright_edge,
            <tile | key, resolver>)
wo:drawRect(upperleft_edge, width, height,
            <tile | key, resolver>)
\end{verbatim}

\subsection*{world floor}
\begin{verbatim}
result = wo:fl(<pos | {x, y} | x,y | obj | group | polist>)
\end{verbatim}

\subsection*{world item}
\begin{verbatim}
result = wo:it(<pos | {x, y} | x,y | obj | group | polist>)
\end{verbatim}

\subsection*{shuffleOxyd}
\begin{verbatim}
wo:shuffleOxyd(rules)
\end{verbatim}

\subsection*{world stone}
\begin{verbatim}
result = wo:st(<pos| {x, y} | x,y | obj | group | polist>)
\end{verbatim}



\section*{Functions}

\subsection*{cond}
\begin{verbatim}
cond(condition, iftrue, iffalse)
\end{verbatim}

\subsection*{fl}
\begin{verbatim}
result = fl(<pos | {x, y} | x,y | obj | group | polist>)
\end{verbatim}

\subsection*{grp}
\begin{verbatim}
grp(<{obj1,obj2, ...} | obj1,obj2, ... | group>)
\end{verbatim}

\subsection*{it}
\begin{verbatim}
result = it(<pos | {x, y} | x,y | obj | group | polist>)
\end{verbatim}

\subsection*{ORI2DIR}
\begin{verbatim}
result = ORI2DIR[orientation]
\end{verbatim}

\subsection*{random}
\begin{verbatim}
result = random(<| n | l,u>)
\end{verbatim}

\subsection*{st}
\begin{verbatim}
result = st(<pos | {x, y} | x,y | obj | group | polist>)
\end{verbatim}



\begin{center}
\rule{0.6\linewidth}{0.25pt} \\
Compiled from Enigma 1.21 reference manual by Raoul
\end{center}

\end{multicols}
\end{document}
